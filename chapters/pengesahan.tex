%lembar pengesahan
\newpage
%\thispagestyle{empty}
\addcontentsline{toc}{chapter}{HALAMAN PENGESAHAN}

\begin{tikzpicture}[remember picture, overlay]
	\node at (current page.center) {\includegraphics[height=10cm]{gambar/logo-ugm-emas.jpeg}};
\end{tikzpicture}

\begin{center}
    \begin{doublespace}
        \textbf{\large \MakeUppercase{HALAMAN PENGESAHAN\\ PRAKTIK INDUSTRI}}
    \end{doublespace}
\end{center}

\begin{center}
    %\textbf{\large \MakeUppercase {\judulid}}
        \textbf{\normalsize \MakeUppercase {\judulid}}
\end{center}

\begin{center}
    Disusun oleh:\\
    \textbf{\penulis}\\
    \textbf{\nim}\\[0.5cm]

    Diajukan untuk memenuhi salah satu syarat memperoleh gelar Sarjana Terapan Teknik pada Program Studi {\prodi} {\departemen} {\fakultas} {\universitas}\\[1cm]
\end{center}

\begin{center}
    \textbf{Hasil dari Praktik Industri ini telah diseminarkan pada:}
\end{center}

\begin{table}[h!]
    \begin{tabular}{p{1cm} p{3cm} p{1cm} l}
        & \textbf{Hari, Tanggal} & \textbf{:} & \\ 
        & \textbf{Pukul}         & \textbf{:} & \\ 
        & \textbf{Tempat}        & \textbf{:} & \\[1cm]
    \end{tabular}
    \label{tabel1}
\end{table}

\begin{center}
    Diterima dan disetujui oleh,
\end{center}

\begin{center}
    Mengetahui, \\[1cm]
\end{center}
\begin{minipage}{0.35\textwidth}
    Ketua Program Studi\\
    Teknologi Rekayasa Elektro\\[2.5cm]
    \underline{\koorprodi}\\
    NIP. \NIPkoorprodi
\end{minipage}%
\hfill
 \begin{minipage}{0.35\textwidth}
    Dosen Pembimbang \\
    Praktik Industri \\[2cm]
    
    \underline{\pembimbing}\\
    NIP. \NIPpembimbing
\end{minipage}